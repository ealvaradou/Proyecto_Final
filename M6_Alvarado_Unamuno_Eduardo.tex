\documentclass[a4paper,11pt]{article}
\usepackage{apacite}
\usepackage[spanish]{babel}
\usepackage[utf8]{inputenc}
\usepackage[dvips]{graphicx}
\DeclareGraphicsExtensions{.png}
\usepackage{multirow}
\usepackage{float}
\graphicspath{{img/}}
\begin{document}
\title{Optimizaci\'on de recursos inform\'aticos con soluciones de escritorios virtuales}
\author{Alvarado Unamuno Eduardo}
\date{\today}
\maketitle
\begin{bf}
\begin{center}
RESUMEN \\
\end{center}
\end{bf}
El objetivo del presente trabajo de investigaci\'on es demostrar 
el ahorro de energi\'ia y costo de la infraestructura a trav\'es 
de una solucion de escritorios virtuales la investigaci\'on parte de los datos proporcionados 
por la Universidad de Guayaquil y tomados experimentalmente, se presenta el an\'alisis de los 
datos recopilados y se concluye que las soluciones de VDI permiten ahorro en el consumo de energ\'ia
por ende disminuci\'on en los gastos por consumo de energia.\\   
\\
\begin{bf}
Palabras claves:
\end{bf}
Virtualizaci\'on, Escritorios Virtuales, Computación Verde. \\
\begin{bf}
\begin{center}
INTRODUCCION\\
\end{center}
\end{bf}
En la actualidad los computadores se han convertido en una herramienta de 
aprendizaje e investigaci'on en los establecimeintos educativos superiores,a a pesar de esto
en algunas instituciones los recursoso son escasos, o presentan discontinuidad
tecnol\'ogica, esto dificulta el proceso de aprendizaje y organizaci\'on de
tareas del estudiante y en la investigaci\'on los docentes no cuentan con
equipos con caracteristicas adecuados para este \'ambito.\\
Desde el punto de vista de los estudiantes, investigadoes y docentes los laboratorios
poseen recursos tecnol\'ogicos limitado, generando inconvenientes al momento
de realizar pr\'acticas tanto en los estudiantes como en los docentes,
lo cual no les permite mantener un control sobre la continuidad de desarrollo de su trabajo,
en las materias que se dictan dichos laboratorios, es as\'i que, en el campo de informática, 
las universidades tienen dificultad en transmitir la instrucci\'on practica a sus estudiantes, debido
a la falta de laboratorios o infraestructuras tecnologicas apropiadas de aprendizaje, ademas las mismas
deben destinar parte de su presupuesto para reemplazo de partes y equipamiento, soporte, mantenimiento y pagos
del consumo de energ\'ia el\'ectrica de la infraestructura tecnol\'ogica de estos laboratorios.\\
Como lo muestra la figura 1, los costos de mantenimiento y administraci\'on generaron el doble de costos de TI comparado
con el costo de la adquisici\'on del servidorr y los costos de energ\'ia y refrigeraci\'on crecieron lo 
suficiente como para acercarse a los costos de adquisici\'on de los servidores a nivel mundial.\\
\\
\\
\begin{figure}[htb]
\centering
   \includegraphics[0cm,0cm][10cm,5cm]{figura-2.png}
   \caption{Comparativo de costos de adquisici\'on, mantenimiento, energ\'ia el\'ectrica y enfriamiento}
\label{Figura 1}
\end{figure}
\begin{bf}
\begin{center}
METODOLOGIA\\
\end{center}
\end{bf}
El procesamiento de datos a lo largo del tiempo ha pasado por varias etapas,
en sus inicios los centros de datos iniciaron procesando datos en grandes computadoras que
trabajaban en forma centralizada, pero requer\'ian una alta inversion a la hora de 
adquirirlas. Luego se desarrollaron computadoras con menor poder de procesamiento, pero m\'as
econ\'omicas y  pequeñas que dominaron en el mercado procesando datos en forma distribuida. El
concepto de procesamiento distribuido, permitio procesar informacion de manera m\'as econ\'omica 
que su antecesor, pero tambi\'en g\'enero inconvenientes como la complejidad en la administraci\'on
y sobre todo subutilizaci\'on de los recursos de cada computadora, lo cual dio cabida a la aplicaci\'on
del concepto de la virtualizaci\'on.\\
La virtualizacion sigue siendo una tendencia tecnológica muy interesante y seg\'un 
Garnert Inc, en un centro de datos actual, casi la mitad de los servidores basados en 
X86 est\'an virtualizadas, entre los motivos que las empresas adoptan la virtualizaci\'on
son la consolidaci\'on de hardware de los servidores, mejora la eficiencia operativa,
optimizaci\'on de recursos limitados y reducci\'on de gastos de funcionameinto la virtualizaci\'on 
de escritorios denominada VDI es similar al concepto de computaci\'on centralizada utilizada
por los mainframe donde los usuarios se conectan a este, a trav\'es de un cliente ligero que es 
limitado en recursos, de igual manera en un entorno de escritorios virtuales, el usuario est\'a conectado
al servidor a trav\'es de un cliente ligero y utilizan las capacidades inform\'aticas de un servidor. Sin
embargo, el usuario tiene la experiencia que maneja su propia instancia de sistema operativo de
escritorio y aplicaciones, pero esa instancia se ejecuta en una maquina virtual alojada en un servidor
fisico \cite{Agrawal2014}.\\
Dadas las ventajas de utilizar los escritorios virtuales, la aplicaci\'on de los escritorios virtuales en
entornos de educaci\'on superior se vuelven muy atractivo, como por ejemplo la implementaci\'on de infraestructura
de escritorios virtuales en laboratorios acad\'emicos \cite{Chroback2014} y \cite{Enrique2016},  dio como resultado despues de 2 años de
la implementaci\'on de la virtualizaci\'on de la infraestructura de escritorios en la Universidad de Economia
en Wroclaw (Polonia) se reitera la eficiencia que promete en cuanto al uso de los recursoso y la oportunidad
de accesder a software especializado desde cualquier dispositivo.\\
\begin{bf}
RESULTADOS\\ 
\end{bf}
Se determin\'o con tomas de valores en sitio el consumo de energ\'ia el\'ectrica de cadadEs una red de ordenadores con acceso restringido que solo pueden utilizar los 
usuarios autorizados de una empresa/institución/grupo y que fomenta el intercambio
de información entre los miembros de una empresa, con el objetivo fundamental de 
asegurar una comunicación fluida entre los empleados o entre empleados y clientes, 
uno de los cuatro laboratorios. A partir de los valores tomados se determin\'o 
el consumo energ\'etico de cada laboratorio y se obtuvo el c\'alculo del valor
a pagar, se considero una jornada acad\'emicoa de 9 horas y 6 di\'as laborables a la 
semana ver tabla 1.\\
La formula utilizada para el calculo de la potencia es $P(kw)=V*I$ donde $V$ es voltaje y $I$ es corriente
\begin{table}[h]
\begin{center}
\begin{tabular}{| c | c | c | c | c |}
\hline
\multicolumn{5}{ |c| }{ Consumo energetico} \\ \hline Unidad & LAB.S1 & LAB.302 & LAB.S2 & LAB.N1 \\ \hline Cantidad PC & 38 &
 26 & 26 & 32 \\ \hline Consumo Kwh & 2.38 & 1.84 & 1.84 & 2.40 \\ \hline Consumo Kwh 1 d\'ia & 21.38 &
16.52 & 16.52 & 21.60 \\ \hline Valor Mensual & 34.47 & 26.64 & 26.64 & 34.82 \\ \hline
\end{tabular}  
\caption {Tabla de consumos energëticos y costos utilizando computadores de escritorios}
\end{center}
\end{table}
\\
\\
\\
\\\bibliographystyle{apacite} 
\bibliography{M6_Alvarado_Unamuno_Eduardo}
\end{document}
